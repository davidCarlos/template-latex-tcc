\part{Fundamentação Teórica}

\chapter{Sistemas de classificação}
Sistemas de classificação são largamente utilizados na indústria e na academia. Estudos em diferentes áreas do conhecimento
lançam mão de algorítmos de aprendizado de máquina para extrair valor de dados que não necessáriamente são totalmente objetivos. Na realidade,
técnicas de aprendizado de máquina, são tolerantes a dados que são imprecisos, parcialmente incorretos ou incertos \cite{Malhotra}. Um primeiro 
exemplo que podemos citar, é a utilização de técnicas de aprendizado de máquina para classificar modulos e classes como passíveis, ou não, a falha\cite{Malhotra}.  A construção de um modelo que represente esse tipo de contexto, leva em consideração métricas de código fonte e dados previos de falhas\cite{Malhotra}. 

A área de saúde também pode se beneficiar de sistemas de recomendação. Uma pesquisa feita pela Universidade da Macedônia,  com pacientes portadores de Parkinson e pessoas sem a doença, permitiu, a partir da construção de um modelo estatístico e um classificador Bayesiano, diagnosticar com uma precisão de 91\%, parcientes com Parkinson\cite{Kotsavasiloglou}. O modelo estatístico definido pela Universidade levou em consideração métricas extraidas a partir do uso de tablets pelos voluntários. Era solicitado que se desenhasse, em um tablet, 11 linhas horizontais da esquerda para a direita, e outras 10 linhas horizontais da direita para esquerda. A partir dessa atividade os pesquisadores extrairam as métricas utilizadas no modelo, uma vez que os voluntários, foram divididos em três grupos distintos, sendo um deles de pacientes com a doença. Vale ressaltar que o estudo foi motivado principalmente pela subjetividade em diagnosticar uma doença como Parkinson, propondo assim um metodo que agregue a futuros diagnósticos.

% TODO: Trazer mais um exemplo de sistemas de classificação

Para construir um sistema de classificação é imprecindivel a definição de um modelo estatístico que represente o sistema complexo em que estamos inseridos, e a utilização de um classificador, que ao ser treinado a partir do modelo, pode vir a classificar dados novos. Dizemos que o modelo representa um sistema complexo, justamente lidar com diversas variáveis distintas, e que podem, estatisticamente falando, terem influência uma sobre as outras.


\chapter{Modelos estatísticos}


\chapter{Classificadores}


\begin{comment}
 1 - Primeiro podemos começar definindo conceitos básicos como categoria, modelo de dados, e 
 quais modelos alternativos ao bayer existem, e porquê utilizamos o bayer.
 
 Definir como um modelo de dados é construido, isso implica em abordar os conceitos estatísticos visto
 nas vídeo aulas do youtube.

 Definir como algorítmo de bayer utiliza o modelo de dados.
\end{comment}
