\chapter*[Introdução]{Introdução}
\addcontentsline{toc}{chapter}{Introdução}

% TODO: Exemplificar melhor sistemas de classificação
De maneira formal, o processo de classificação de um conjunto de dados pode ocorrer de diferentes maneiras. De maneira geral, classificação é uma tarefa básica em mineração de dados que requer a construção de de um classificador, ou seja, uma função que atribui uma categoria aos dados observados, descritos por um conjunto de variáveis \cite{Taheri}. Um sistema de classificação, é composto por duas partes. O modelo de dados, que representa as variáveis do mundo real que estamos tratando, e o próprio classificador, que recebe como entrada este modelo definido previamente, atribuindo assim aos dados processados, categorias do domínio, ou seja, classificando-os.

 Uma categoria, depende do contexto a qual esses dados pertencem, e está diretamente relacionada, ao modelo de dados construído. Contextos como disgnósticos de doenças, processamento e reconhecimento de imagens, fazem uso extensivo de algorítimos de classificação. O processo de categorização irá receber o modelo de dados como entrada, modelo esse que irá, de maneira simplificada, representar as variáveis do sistema complexo que estamos lidando. É nesse ambito do processo de classificação que a estatística é utilizada.

 O fato de estarmos lidando com sistemas complexos, que podem vir a ter diversas variáveis diferentes, e que podem sofrer influência entre si, é que aplicamos os conceitos da computação, ao lançarmos mão de algorítmos que nos permitam aplicar um modelo de dados a uma entrada desconhecida, e termos como saída, uma categorização para esses dados.

Problemas que envolvem aprendizado de máquina são bastante heterogêneos, e diversas estratégias, tanto do ponto de vista de algorítmos, quando de modelos estatísticos foram propostas na literatura. Isso faz com que diferentes domínios utilizem aprendizado de máquina
em suas aplicações. Alguns exemplos:

\begin{itemize}
\item Análise de imagens para identificar formas e estruturas distintas, como para análises médicas e reconhecimento de impressões digitais\cite{Louridas}.

\item Uso de big data para criação de regras para análise de dados. Utilizado na área de marketing e promoções de vendas\cite{Louridas}.

\item Reconhecimento de padrões para analizar código fonte em busca de fraquezas, como más práticas e falhas críticas\cite{Louridas}.

\item Criação de eurísticas de segurança para refinar padrões de ataque, protegendo assim portas e redes \cite{Louridas}.
\end{itemize}

 % TODO: Falar um pouco sobre a incerteza de qualidade presente em projetos opensource, para então linkar com os conceitos do Debian

Segundo \cite{Zacchiroli}, pacotes são abstrações defininindo a granularidade em que usuários podem atuar(adicionar, remover, atualizar, etc) em um software disponível. Ainda
segundo \cite{Zacchiroli} uma distribuição é uma coleção de pacotes mantidos de maneira coerente.
O Debian GNU/Linux, é uma distribuição do sistema operacional Linux, e os vários pacotes que são executados sobre ele\cite{Debian}.
O Linux é um clone do sistema operacional Unix, escrito do zero por por Linus Torvalds com a ajuda de um time de programadores distribuidos pelo internet \cite{Linux}.

O Debian Gnu/Linux é:
        \begin{itemize}
        \item \textbf{funcional}: Debian inclui 49.096 pacotes no presente momento. O usuário
        pode escolher qual pacote instalar;O Debian provê uma ferramenta para esse proposito.\cite{Debian}

        \item \textbf{livre para ser usado e redistribuido}: Não há um consórcio ou pagamento exigido para participar na sua distribuição e desenvolvimento.
        Todos os pacotes que são parte do Debian GNU/Linux são de distribuição livre, usualmente sob os termos especificados pela licença GPL.\cite{Debian}

        \item \textbf{dinamico}: Com cerca de 1033 voluntários contribuindo constantemente com código novo e melhorado, Debian evolui rapidamente.\cite{Debian}
        \end{itemize}


        A estrutura de um pacote Debian é formalmente estabelecida por um documento chamado \textit {Debian Policy Manual}. Abordaremos esse documento de maneira mais profunda ao longo deste trabalho.Além de estabelecer a política que rege a distribuição como um todo,  esse manual também se atem a definir a interface do sistema de gerenciamento de pacotes em que os desenvolvedores tem de estar familizarizados \cite{Guide}. É com \textit {Debian Policy Manual} que iremos definir o que deve existir dentro de um pacote Debian. Além deste manual também estaremos utilizando como insumo para essa pesquisa o documento cchamado \textit{Guide for Debian Maintainers}. Nesse guia prático, Osamu Aoki demonstra como um pacote Debian é construído, quais ferramentas são utilizadas, e como a política definida pelo \textit{Debian Policy Manual} deve ser implementada.


        Analisar se um pacote já existente é bom ou não no Debian, é uma tarefa que envolve avaliar aplicações como o BTS, que informa uma série de informações sobre o pacote, dados do lintian, ferramenta de análise estática para pacotes Debian, e também o DebCi, infraestrutura de integração contínua para pacotes Debian. Essa avaliação é feita por qualquer um que queira saber qual a situação de um pacote no Debian. Criar um sistema que possa classificar pacotes Debian de maneira coerente, e que gere uma saída que possa ser usada em outros contextos, é a motivação principal desse trabalho.

\section{Objetivos}

        Essa pesquisa tem por objetivo geral desenvolver um sistema de classificação para pacotes Debian, de maneira que dado um pacote
        como entrada, possamos avaliá-lo estatistamente, dando assim um parecer sobre a sua qualidade do ponto de vista da política para pacotes Estabelicida pelo Debian.
        Este objetivo geral está atrelado a algumas questões de pesquisa, que serão respondidas ao longo desta pesquisa:
        \begin{itemize}
                \item Quais variáveis são mais propicias para classificar um pacote debian dentro de uma escala previamente definida?
                \item É possível extrair da base de pacotes, uma modelo de dados viável, que possa ser utilizado pelos algorítmos de classificação abordados na literatura?
        \end{itemize}

        Para alcançar este objetivo geral, os seguintes objetivos tecnologicos serão contemplados:
                \begin{enumerate}
                        \item Evolução da base de dados do software Pet.
                        \item Construção de um modelo de dados a partir da base de dados do software Pet.
                        \item Coleta de dados a partir de um conjunto de pacotes, para que os parâmetros estatisticos utilizados no modelo de dados sejam minimamente válidos.
                        \item Implementação de um algorítmo de classificação, utilizando o modelo de dados contruído a partir do Pet
                \end{enumerate}

\section{Organização do Trabalho}

        Esta pesquisa será estruturada em três capitulos. No capitulo 2 abordaremos a estrutura atual do Pet e analizaremos as apis externas que iremos utilizar para evoluir sua base de dados.
        No capitulo 3 faremos uma revisão bibliográfica sobre o estado da arte de sistemas de classificação, os algorítmos e abordagens(computacionais e estatísticas) utilizadas em sistema dessa categoria. No capitulo 4 abordaremos a metodologia para a implementação do algorítmo de classificação para pacotes Debian. No capitulo 5 discutiremos os resultados alcançados e as demais considerações sobre o trabalho.
