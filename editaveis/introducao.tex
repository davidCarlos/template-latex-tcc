\chapter*[Introdução]{Introdução}
\addcontentsline{toc}{chapter}{Introdução}

Problemas que envolvem aprendizado de máquina são bastante heterogêneos, e diversas estratégias, tanto do ponto de vista de algorítmos, quando de modelos estatísticos foram propostas na literatura. Isso faz com que diferentes domínios utilizem aprendizado de máquina
em suas aplicações. Alguns exemplos:

\begin{itemize}
\item Análise de imagens para identificar formas e estruturas distintas, como para análises médicas e reconhecimento de impressões digitais\cite{Louridas}.

\item Uso de big data para criação de regras para análise de dados. Utilizado na área de marketing e promoções de vendas\cite{Louridas}.

\item Reconhecimento de padrões para analizar código fonte em busca de fraquezas, como más práticas e falhas críticas\cite{Louridas}.

\item Criação de eurísticas de segurança para refinar padrões de ataque, protegendo assim portas e redes \cite{Louridas}.
\end{itemize}

 % TODO: Falar um pouco sobre a incerteza de qualidade presente em projetos opensource, para então linkar com os conceitos do Debian


        Analisar se um pacote já existente é bom ou não no Debian, é uma tarefa que envolve avaliar aplicações como o BTS, que informa uma série de informações sobre o pacote, dados do lintian, ferramenta de análise estática para pacotes Debian, e também o DebCi, infraestrutura de integração contínua para pacotes Debian. Essa avaliação é feita por qualquer um que queira saber qual a situação de um pacote no Debian. Criar um sistema que possa classificar pacotes Debian de maneira coerente, e que gere uma saída que possa ser usada em outros contextos, é a motivação principal desse trabalho.

\section{Objetivos}

        Essa pesquisa tem por objetivo geral desenvolver um sistema de inferência para pacotes Debian, de maneira que dado o código fonte de um pacote, possamos avaliar estatísticamente, de que maneira o tamanho do projeto, a linguagem de programação utilizada e número de contribuidores, influênciam na coleta de fraquezas de software, por parte de ferramentas de análise estática.

        Este objetivo geral está atrelado a algumas questões de pesquisa, que serão respondidas ao longo desta pesquisa:
        \begin{itemize}
                \item De que maneira variáveis como tamanho do projeto, número de contribuidores e linguagem de programação influênciam de alguma forma em fraquezas reportadas por ferramentas de análise estática?
                \item É possível construir um modelo estatístico que represente a relação entre fraquezas de software catalogadas e as variáveis como tamanho do projeto, número de contribuidores e linguagem de programação?
        \end{itemize}

        Para alcançar este objetivo geral, os seguintes objetivos tecnologicos serão contemplados:
                \begin{enumerate}
                        \item Construção de um modelo de dados a partir dos dados disponibilizados pela ferramenta $Sate$.
                        \item Configuração e instalação da ferramenta $Debile$.
                        \item Construção de uma base de dados de pacotes debian a partir da ferramenta $Debile$.
                        \item Aplicação do modelo estatístico criado, na base de dados estabelecido com a ferramenta $Debile$.
                \end{enumerate}

\section{Organização do Trabalho}

        Esta pesquisa será estruturada em três capitulos. No capitulo 2 faremos uma revisão bibliográfica sobre o estado da arte do aprendizado de máquina, os algorítmos e abordagens(computacionais e estatísticas) utilizadas em sistema dessa categoria. No capitulo 3 abordaremos a metodologia para a implementação do algorítmo de inferência para pacotes Debian. No capitulo 4 discutiremos os resultados alcançados e as demais considerações sobre o trabalho.
