\chapter*[Introdução]{Introdução}
\addcontentsline{toc}{chapter}{Introdução}

-> Questões de quailidade/metricas na ENG de SOFT

Segundo \cite{Zacchiroli}, pacotes são abstrações defininindo a granularidade em que
usuários podem atuar(adicionar, remover, atualizar, etc) em um software disponível. Ainda
segundo \cite{Zacchiroli} uma distribuição é uma coleção de pacotes mantidos de maneira coerente.
O Debian GNU/Linux, é uma distribuição do sistema operacional Linux, e os vários pacotes que são executados sobre ele\cite{Debian}.

O Linux é um clone do sistema operacional Unix, escrito do zero por por Linus Torvalds com a ajuda de um time de programadores distribuidos pelo internet \cite{Linux}.

A distribuição Debian, atualmente, pode ser utilizada com outros kernels(Hurd, BSD), mas o foco dessa pesquisa será com o Debian GNU/Linux.

O Debian Gnu/Linux é:
        \begin{itemize}
        \item \textbf{funcional}: Debian inclui mais de quarenta e dois mil, quinhentos e cinquenta e um pacotes no presente momento. O usuário
        pode escolher qual pacote instalar;O Debian provê uma ferramenta para esse proposito.\cite{Debian}

        \item \textbf{livre para ser usado e redistribuido}: Não há um consórcio ou pagamento exigido para participar na sua distribuição e desenvolvimento.
        Todos os pacotes que são parte do Debian GNU/Linux são de distribuição livre, usualmente sob os termos especificados pela licença GPL.\cite{Debian}

        \item \textbf{dinamico}: Com cerca de mil e trinta e três voluntários contribuindo constantemente com código novo e melhorado, Debian evolui rapidamente.\cite{Debian}
        \end{itemize}

Nesta pesquisa iremos utlizar a distribuição Debian e seus repositórios oficiais, além do software AppRecommender.

O AppRecommender 



-> Conceituar aplicações diversas e como elas viram pacotes

Com essa conceituação podemos inferir, que pacotes são a base para qualquer distribuição que utilize


-> Conceituar kernel

-> Conceituar o Debian

-> Falar sobre sistemas de recomendação

-> Falar sobre integração contínua

