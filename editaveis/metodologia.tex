\chapter{Metodologia}
\addcontentsline{toc}{chapter}{Metodologia}

\section{Classificação da Pesquisa}

Neste capitulo iremos definir a metodologia estabelecida para essa pesquisa, bem como
a fonte dos dados utilizados, os trabalhos relacionados, de onde buscamos o referêncial teórico,
além da classificação da pesquisa em si. 

Esta pesquisa pode ser classificada, do ponto de vista de sua natureza, como aplicada, uma vez que a apartir dos conhecimentos bibliográficos,
iremos propor uma ferramenta para classificação de pacotes debian. Do ponto de vista da forma de abordagem, podemos classificá-la como 
quantitativa, pois a avaliação dos pacotes será feita por técnicas estatísticas.

Alinhada os objetivos do trabalho, esta pesquisa é exploratória, e se constroi a apartir dos estudos realizados na área de aprendizado
de máquina.

Do ponto de vista dos procedimentos técnicos, esta pesquisa é bibliográfica e experimental.

\section{Trabalhos relacionados}

A área de aprendizado de máquina é bastante vasta, e aplicável a diferentes domínios. 
Durante o levantamento bibliográfico encontrou-se diversos sistemas de classificação que possuem certa similaridade com o que será proposto por essa pesquisa.
Um primeiro trabalho relacionado que podemos citar, é a utilização de técnicas de aprendizado de máquina para classificar modulos e classes como passíveis, ou não, a falha\cite{Malhotra}.  
A construção de um modelo que represente esse tipo de contexto, leva em consideração métricas de código fonte e dados previos de falhas\cite{Malhotra}.

A área de saúde também pode se beneficiar de sistemas de recomendação. Uma pesquisa feita pela Universidade da Macedônia,  com pacientes portadores de Parkinson e pessoas sem a doença, permitiu, a partir da construção de um modelo estatístico e um classificador Bayesiano, diagnosticar com uma precisão de 91\%, parcientes com Parkinson\cite{Kotsavasiloglou}. O modelo estatístico definido pela Universidade levou em consideração métricas extraidas a partir do uso de tablets pelos voluntários. Era solicitado que se desenhasse, em um tablet, 11 linhas horizontais da esquerda para a direita, e outras 10 linhas horizontais da direita para esquerda. A partir dessa atividade os pesquisadores extrairam as métricas utilizadas no modelo, uma vez que os voluntários, foram divididos em três grupos distintos, sendo um deles de pacientes com a doença. Vale ressaltar que o estudo foi motivado principalmente pela subjetividade em diagnosticar uma doença como Parkinson, propondo assim um metodo que agregue a futuros diagnósticos.

Para o domínio de pacotes debian aliado ao uso de um modelo grafico probabilístico que tenha como fonte de dados estes pacotes, podemos citar a Aplicação \textit{AppRecommender}. Essa aplicação recomenda pacotes Debian para o usuário,
a partir da criação de um perfil, estabelecido utilizando os pacotes manualmente instalados na máquina. 
O processo de recomendação faz uso do algorítmo de bayes, e a fase de treinamento do algorítmo foi feita coletando dados de vários usuários da distribuição Debian.

\section{Coleta de dados}

\subsubsection{Ferramentas}

\subsection{}

\section{Modelo de dados}

\section{Análise Estatística}

\section{Limitações do trabalho}


