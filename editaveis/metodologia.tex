\chapter{Metodologia}
\addcontentsline{toc}{chapter}{Metodologia}

\section{Classificação da Pesquisa}

Neste capitulo iremos definir a metodologia estabelecida para essa pesquisa, bem como
a fonte dos dados utilizados, os trabalhos relacionados, de onde buscamos o referêncial teórico,
além da classificação da pesquisa em si.

Esta pesquisa pode ser classificada, do ponto de vista de sua natureza, como aplicada, uma vez que a apartir dos conhecimentos bibliográficos,
iremos propor um modelo probabilistico para inferência de fraquezas de software, a partir de variáveis como a linguagem do projeto e o seu tamanho(medido por sloccount). Do ponto de vista da forma de abordagem, podemos classificá-la como
quantitativa, pois a inferência realizada utilizando os pacotes debian será feita por técnicas estatísticas.

Alinhada aos objetivos do trabalho, esta pesquisa é exploratória, e se constroi a apartir dos estudos realizados na área de aprendizado
de máquina.

Do ponto de vista dos procedimentos técnicos, esta pesquisa é bibliográfica e experimental.

\section{Trabalhos relacionados}

A área de aprendizado de máquina é bastante vasta, e aplicável a diferentes domínios.
Durante o levantamento bibliográfico encontrou-se diversos sistemas que fazem uso de aprendizado de máquina e que possuem certa similaridade com o que será proposto por essa pesquisa.
Um primeiro trabalho relacionado que podemos citar, é a utilização de técnicas de aprendizado de máquina para classificar modulos e classes como passíveis, ou não, a falha\cite{Malhotra}.
A construção de um modelo que represente esse tipo de contexto, leva em consideração métricas de código fonte e dados previos de falhas\cite{Malhotra}.

A área de saúde também pode se beneficiar de sistemas de recomendação. Uma pesquisa feita pela Universidade da Macedônia,  com pacientes portadores de Parkinson e pessoas sem a doença, permitiu, a partir da construção de um modelo estatístico e um classificador Bayesiano, diagnosticar com uma precisão de 91\%, parcientes com Parkinson\cite{Kotsavasiloglou}. O modelo estatístico definido pela Universidade levou em consideração métricas extraidas a partir do uso de tablets pelos voluntários. Era solicitado que se desenhasse, em um tablet, 11 linhas horizontais da esquerda para a direita, e outras 10 linhas horizontais da direita para esquerda. A partir dessa atividade os pesquisadores extrairam as métricas utilizadas no modelo, uma vez que os voluntários, foram divididos em três grupos distintos, sendo um deles de pacientes com a doença. Vale ressaltar que o estudo foi motivado principalmente pela subjetividade em diagnosticar uma doença como Parkinson, propondo assim um metodo que agregue a futuros diagnósticos.

Para o domínio de pacotes debian aliado ao uso de um modelo grafico probabilístico que tenha como fonte de dados estes pacotes, podemos citar a Aplicação \textit{AppRecommender}. Essa aplicação recomenda pacotes Debian para o usuário,
a partir da criação de um perfil, estabelecido utilizando os pacotes manualmente instalados na máquina.
O processo de recomendação faz uso do algorítmo de bayes, e a fase de treinamento do algorítmo foi feita coletando dados de vários usuários da distribuição Debian.

\section{Coleta de dados}
Os dados que serão utilizados nessa pesquisa, serão em parte dados primários e em parte dados secundários. Os dados secundários serão utilizados na fase de treinamento do aprendizado de máquina, buscando encontrar o melhor modelo estatístico que traduza o domínio com que estamos lidando. Os dados primários serão coletados a partir da implantação de algumas ferramentas, que farão parte da construção dessa pesquisa.

\subsection{Sate}
Os dados secundários foram obtidos por meio do projeto $Sate$. Nesse projeto, várias ferramentas de análise estática foram executadas em diferentes projetos. O resultado obtido por essas ferramentas foi agrupado em uma linguagem única, e disponiblizado em formato $xml$ na internet. Essa padronização foi feita pelo fato de que cada ferramenta utilizada na fase de coleta, gerava uma saída diferente, dificultando a análise feita pelos pesquisadores. O projeto $Sate$ durou de 2008 até 2012, sendo que a cada ano de projeto, um conjunto de dados análisados e um relatório da pesquisa foram disponibilizados(O ano de 2011 não possui dados e também não possui relatório).

\newpage
\begin{figure}[h]
	\centering
	\label{sate_schema}
        \includegraphics[scale=0.42]{figuras/sate_schema.eps}
	\caption{O padrão xml que as ferramentas que participaram do projeto adotaram}
\end{figure}

A partir do uso do esquema definido na figura~\ref{sate_schema}, foi possível definir um processo de análise do relatório
gerado pelas diferentes ferramentas escolhidas. Um exemplo de relatório padronizado da ferramenta $cppcheck$, utilizada no projeto.

\begin{figure}[h]
	\centering
	\label{cppcheck}
        \includegraphics[scale=0.42]{figuras/cppcheck.eps}
	\caption{Exemplo de relatório da ferramenta cppcheck}
\end{figure}
\newpage

A primeira etapa do projeto foi a submissão dos relatórios gerados pelas ferramentas, utilizando o esquema da figura~\ref{sate_schema}. A segunda etapa foi a análise dos relatórios por parte de pesquisadores do $Mitre$ em conjunto com os envolvidos no projeto $Sate$. Uma das abordagens utilizadas nessa análise envolvia relacionar os problemas relatados pelas ferramentas a $CWE's$ catalogadas. Essa abordagem permitiu chegar as seguintes caracteríticas de uma $CWE$:

\begin{itemize}
        \item Pode ser associada com mais de uma $CWE$ ($CWE$ composta).
        \item Pode estar relacionada a mais de uma chamada no código fonte.
        \item Pode possuir mais de um fluxo de dados.
\end{itemize}

Os dados coletados por $Sate$, nos permite afirmar que: \emph{Estima-se que, apenas entre 1/8 e 1/3  de todas as fraquezas coletadas são simples.}

\subsection{Debile}

\subsection{Sloccount}
